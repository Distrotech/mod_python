\chapter{Apache Configuration Directives\label{directives}}

\section{Request Handlers\label{dir-handlers}}

\subsection{Python*Handler Directive Syntax\label{dir-handlers-syn}}
\index{Python*Handler Syntax}

All request handler directives have the following syntax: 

\code{Python*Handler \emph{handler [handler ...] [ | .ext [.ext ...] ] } }

Where \var{handler} is a callable object that accepts a single
argument - request object, and \var{.ext} is a file extension.

Multiple handlers can be specified on a single line, in which case
they will be called sequentially, from left to right. Same handler
directives can be specified multiple times as well, with the same
result - all handlers listed will be executed sequentially, from first
to last. If any handler in the sequence returns a value other than
\code{apache.OK}, then execution of all subsequent handlers is aborted.

The list of handlers can optionally be followed by a \code{|} followed
by one or more file extensions. This would restrict the execution of
the handler to those file extensions only. This feature only works for
handlers executed after the trans phase.

A \emph{handler} has the following syntax: 

\code{module[::object]}

Where \var{module} can be a full module name (package dot notation is
accepted), and the optional \var{object} is the name of an object
inside the module.

Object can also contain dots, in which case it will be resolved from
left to right. During resolution, if mod_python encounters an object
of type \code{<class>}, it will try instantiating it passing it a single
argument, a request object.

If no object is specified, then it will default to the directive of
the handler, all lower case, with the word \samp{python}
removed. E.g. the default object for PythonAuthenHandler would be
authenhandler.

Example: 

\begin{verbatim}
  PythonAuthzHandler mypackage.mymodule::checkallowed
\end{verbatim}

For more information on handlers, see Overview of a Handler.

Side note: The \samp{::} was chosen for performance reasons. In order for
Python to use objects inside modules, the modules first need to be
imported. Having the separator as simply a \samp{.}, would considerably
complicate process of sequentially evaluating every word to determine
whether it is a package, module, class etc. Using the (admittedly
un-Python-like) \samp{::} takes the time consuming work of figuring out
where the module part ends and the object inside of it begins away
from mod_python resulting in a modest performance gain.

\subsection{PythonPostReadRequestHandler\label{dir-handlers-prrh}}
\index{PythonPostReadRequestHandler}

\strong{\citetitle[http://httpd.apache.org/docs-2.0/mod/directive-dict.html#Syntax]{Syntax:}}
\emph{Python*Handler Syntax}\\
\citetitle[http://httpd.apache.org/docs-2.0/mod/directive-dict.html#Context]{Context:}
server config, virtual host\\
\citetitle[http://httpd.apache.org/docs-2.0/mod/directive-dict.html#Override]{Override:}
not None\\
\citetitle[http://httpd.apache.org/docs-2.0/mod/directive-dict.html#Module]{Module:}
mod_python.c

This handler is called after the request has been read but before any
other phases have been processed. This is useful to make decisions
based upon the input header fields.

\begin{notice}
When this phase of the request is processed, the URI has not yet
been translated into a path name, therefore this directive could never
be executed by Apache if it could specified within \code{<Directory>},
\code{<Location>}, \code{<File>} directives or in an \file{.htaccess}
file. The only place this directive is allowed is the main
configuration file, and the code for it will execute in the main
interpreter. And because this phase happens before any identification
of the type of content being requested is done (i.e. is this a python
program or a gif?), the python routine specified with this handler
will be called for \emph{ALL} requests on this server (not just python
programs), which is an important consideration if performance is a
priority.
\end{notice}

\indexii{phase}{order} The handlers below are documented in order in
which phases are processed by Apache.

\subsection{PythonTransHandler\label{dir-handlers-th}}
\index{PythonTransHandler}

\strong{\citetitle[http://httpd.apache.org/docs-2.0/mod/directive-dict.html#Syntax]{Syntax:}}
\emph{Python*Handler Syntax}\\
\citetitle[http://httpd.apache.org/docs-2.0/mod/directive-dict.html#Context]{Context:}
server config, virtual host\\
\citetitle[http://httpd.apache.org/docs-2.0/mod/directive-dict.html#Override]{Override:}
not None\\
\citetitle[http://httpd.apache.org/docs-2.0/mod/directive-dict.html#Module]{Module:}
mod_python.c

This handler gives allows for an opportunity to translate the URI into
an actual filename, before the server's default rules (Alias
directives and the like) are followed.

\begin{notice}
  At the time when this phase of the request is being processed, the
  URI has not been translated into a path name, therefore this
  directive will never be executed by Apache if specified within
  \code{<Directory>}, \code{<Location>}, \code{<File>} directives or
  in an \file{.htaccess} file. The only place this can be specified is
  the main configuration file, and the code for it will execute in the
  main interpreter.
\end{notice}

\subsection{PythonHeaderParserHandler\label{dir-handlers-hph}}
\index{PythonHeaderParserHandler}

\strong{\citetitle[http://httpd.apache.org/docs-2.0/mod/directive-dict.html#Syntax]{Syntax:}}
\emph{Python*Handler Syntax}\\
\citetitle[http://httpd.apache.org/docs-2.0/mod/directive-dict.html#Context]{Context:}
server config, virtual host, directory, htaccess\\
\citetitle[http://httpd.apache.org/docs-2.0/mod/directive-dict.html#Override]{Override:}
not None\\
\citetitle[http://httpd.apache.org/docs-2.0/mod/directive-dict.html#Module]{Module:}
mod_python.c

This handler is called to give the module a chance to look at the
request headers and take any appropriate specific actions early in the
processing sequence.

\subsection{PythonInitHandler\label{dir-handlers-pih}}
\index{PythonInitHandler}

\strong{\citetitle[http://httpd.apache.org/docs-2.0/mod/directive-dict.html#Syntax]{Syntax:}}
\emph{Python*Handler Syntax}\\
\citetitle[http://httpd.apache.org/docs-2.0/mod/directive-dict.html#Context]{Context:}
server config, virtual host, directory, htaccess\\
\citetitle[http://httpd.apache.org/docs-2.0/mod/directive-dict.html#Override]{Override:}
not None\\
\citetitle[http://httpd.apache.org/docs-2.0/mod/directive-dict.html#Module]{Module:}
mod_python.c

This handler is the first handler called in the request processing
phases that is allowed both inside and outside \file{.htaccess} and
directory.

This handler is actually an alias to two different handlers. When
specified in the main config file outside any directory tags, it is an
alias to \code{PostReadRequestHandler}. When specified inside directory
(where \code{PostReadRequestHandler} is not allowed), it aliases to
\code{PythonHeaderParserHandler}.

\emph{(This idea was borrowed from mod_perl)}

\subsection{PythonAccessHandler\label{dir-handlers-ach}}
\index{PythonAccessHandler}

\strong{\citetitle[http://httpd.apache.org/docs-2.0/mod/directive-dict.html#Syntax]{Syntax:}}
\emph{Python*Handler Syntax}\\
\citetitle[http://httpd.apache.org/docs-2.0/mod/directive-dict.html#Context]{Context:}
server config, virtual host, directory, htaccess\\
\citetitle[http://httpd.apache.org/docs-2.0/mod/directive-dict.html#Override]{Override:}
not None\\
\citetitle[http://httpd.apache.org/docs-2.0/mod/directive-dict.html#Module]{Module:}
mod_python.c

This routine is called to check for any module-specific restrictions
placed upon the requested resource.

For example, this can be used to restrict access by IP number. To do
so, you would return \code{HTTP_FORBIDDEN} or some such to indicate
that access is not allowed.

\subsection{PythonAuthenHandler\label{dir-handlers-auh}}
\index{PythonAuthenHandler}

\strong{\citetitle[http://httpd.apache.org/docs-2.0/mod/directive-dict.html#Syntax]{Syntax:}}
\emph{Python*Handler Syntax}\\
\citetitle[http://httpd.apache.org/docs-2.0/mod/directive-dict.html#Context]{Context:}
server config, virtual host, directory, htaccess\\
\citetitle[http://httpd.apache.org/docs-2.0/mod/directive-dict.html#Override]{Override:}
not None\\
\citetitle[http://httpd.apache.org/docs-2.0/mod/directive-dict.html#Module]{Module:}
mod_python.c

This routine is called to check the authentication information sent
with the request (such as looking up the user in a database and
verifying that the [encrypted] password sent matches the one in the
database).

To obtain the username, use \code{req.user}. To obtain the password
entered by the user, use the \code{req.get_basic_auth_pw()} function.

A return of \code{apache.OK} means the authentication succeeded. A
return of \code{apache.HTTP_UNAUTHORIZED} with most browser will bring
up the password dialog box again. A return of
\code{apache.HTTP_FORBIDDEN} will usually show the error on the
browser and not bring up the password dialog
\code{again. HTTP_FORBIDDEN} should be used when authentication
succeeded, but the user is not permitted to access a particular URL.

An example authentication handler might look like this: 

\begin{verbatim}
def authenhandler(req):

    pw = req.get_basic_auth_pw()
    user = req.user     
    if user == "spam" and pw == "eggs":
        return apache.OK
    else:
        return apache.HTTP_UNAUTHORIZED
\end{verbatim}    

\begin{notice}
  \code{req.get_basic_auth_pw()} must be called prior to using the
  \code{req.user} value. Apache makes no attempt to decode the
  authentication information unless \code{req.get_basic_auth_pw()} is called.
\end{notice}

\subsection{PythonAuthzHandler\label{dir-handlers-auzh}}
\index{PythonAuthzHandler}

\strong{\citetitle[http://httpd.apache.org/docs-2.0/mod/directive-dict.html#Syntax]{Syntax:}}
\emph{Python*Handler Syntax}\\
\citetitle[http://httpd.apache.org/docs-2.0/mod/directive-dict.html#Context]{Context:}
server config, virtual host, directory, htaccess\\
\citetitle[http://httpd.apache.org/docs-2.0/mod/directive-dict.html#Override]{Override:}
not None\\
\citetitle[http://httpd.apache.org/docs-2.0/mod/directive-dict.html#Module]{Module:}
mod_python.c

This handler runs after AuthenHandler and is intended for checking
whether a user is allowed to access a particular resource. But more
often than not it is done right in the AuthenHandler.

\subsection{PythonTypeHandler\label{dir-handlers-tph}}
\index{PythonTypeHandler}

\strong{\citetitle[http://httpd.apache.org/docs-2.0/mod/directive-dict.html#Syntax]{Syntax:}}
\emph{Python*Handler Syntax}\\
\citetitle[http://httpd.apache.org/docs-2.0/mod/directive-dict.html#Context]{Context:}
server config, virtual host, directory, htaccess\\
\citetitle[http://httpd.apache.org/docs-2.0/mod/directive-dict.html#Override]{Override:}
not None\\
\citetitle[http://httpd.apache.org/docs-2.0/mod/directive-dict.html#Module]{Module:}
mod_python.c

This routine is called to determine and/or set the various document
type information bits, like Content-type (via \code{r->content_type}),
language, et cetera.

\subsection{PythonFixupHandler\label{dir-handlers-fuh}}
\index{PythonFixupHandler}

\strong{\citetitle[http://httpd.apache.org/docs-2.0/mod/directive-dict.html#Syntax]{Syntax:}}
\emph{Python*Handler Syntax}\\
\citetitle[http://httpd.apache.org/docs-2.0/mod/directive-dict.html#Context]{Context:}
server config, virtual host, directory, htaccess\\
\citetitle[http://httpd.apache.org/docs-2.0/mod/directive-dict.html#Override]{Override:}
not None\\
\citetitle[http://httpd.apache.org/docs-2.0/mod/directive-dict.html#Module]{Module:}
mod_python.c

This routine is called to perform any module-specific fixing of header
fields, et cetera. It is invoked just before any content-handler.

\subsection{PythonHandler\label{dir-handlers-ph}}
\index{PythonHandler}

\strong{\citetitle[http://httpd.apache.org/docs-2.0/mod/directive-dict.html#Syntax]{Syntax:}}
\emph{Python*Handler Syntax}\\
\citetitle[http://httpd.apache.org/docs-2.0/mod/directive-dict.html#Context]{Context:}
server config, virtual host, directory, htaccess\\
\citetitle[http://httpd.apache.org/docs-2.0/mod/directive-dict.html#Override]{Override:}
not None\\
\citetitle[http://httpd.apache.org/docs-2.0/mod/directive-dict.html#Module]{Module:}
mod_python.c

This is the main request handler. Many applications will only provide
this one handler.

\subsection{PythonLogHandler\label{dir-handlers-plh}}
\index{PythonLogHandler}

\strong{\citetitle[http://httpd.apache.org/docs-2.0/mod/directive-dict.html#Syntax]{Syntax:}}
\emph{Python*Handler Syntax}\\
\citetitle[http://httpd.apache.org/docs-2.0/mod/directive-dict.html#Context]{Context:}
server config, virtual host, directory, htaccess\\
\citetitle[http://httpd.apache.org/docs-2.0/mod/directive-dict.html#Override]{Override:}
not None\\
\citetitle[http://httpd.apache.org/docs-2.0/mod/directive-dict.html#Module]{Module:}
mod_python.c

This routine is called to perform any module-specific logging
activities. 

\subsection{PythonCleanupHandler\label{dir-handlers-pch}}
\index{PythonCleanupHandler}

\strong{\citetitle[http://httpd.apache.org/docs-2.0/mod/directive-dict.html#Syntax]{Syntax:}}
\emph{Python*Handler Syntax}\\
\citetitle[http://httpd.apache.org/docs-2.0/mod/directive-dict.html#Context]{Context:}
server config, virtual host, directory, htaccess\\
\citetitle[http://httpd.apache.org/docs-2.0/mod/directive-dict.html#Override]{Override:}
not None\\
\citetitle[http://httpd.apache.org/docs-2.0/mod/directive-dict.html#Module]{Module:}
mod_python.c

This is the very last handler, called just before the request object
is destroyed by Apache.

Unlike all the other handlers, the return value of this handler is
ignored. Any errors will be logged to the error log, but will not be
sent to the client, even if PythonDebug is On.

This handler is not a valid argument to the \code{rec.add_handler()}
function. For dynamic clean up registration, use
\code{req.register_cleanup()}.

Once cleanups have started, it is not possible to register more of
them. Therefore, \code{req.register_cleanup()} has no effect within this
handler.

Cleanups registered with this directive will execute \emph{after} cleanups
registered with \code{req.register_cleanup()}.

\section{Filters\label{dir-filter}}

\subsection{PythonInputFilter\label{dir-filter-if}}
\index{PythonInputFilter}

\strong{\citetitle[http://httpd.apache.org/docs-2.0/mod/directive-dict.html#Syntax]{Syntax:}}
PythonInputFilter handler name\\
\citetitle[http://httpd.apache.org/docs-2.0/mod/directive-dict.html#Context]{Context:}
server config\\
\citetitle[http://httpd.apache.org/docs-2.0/mod/directive-dict.html#Module]{Module:}
mod_python.c

Registers an input filter \var{handler} under name
\var{name}. \var{Handler} is a module name optionally followed
\code{::} and a callable object name. If callable object name is
omitted, it will default to \samp{inputfilter}. \var{Name} is the name under
which the filter is registered, by convention filter names are usually
in all caps.

To activate the filter, use the \code{AddInputFilter} directive.

\subsection{PythonOutputFilter\label{dir-filter-of}}
\index{PythonOutputFilter}

\strong{\citetitle[http://httpd.apache.org/docs-2.0/mod/directive-dict.html#Syntax]{Syntax:}}
PythonOutputFilter handler name\\
\citetitle[http://httpd.apache.org/docs-2.0/mod/directive-dict.html#Context]{Context:}
server config\\
\citetitle[http://httpd.apache.org/docs-2.0/mod/directive-dict.html#Module]{Module:}
mod_python.c

Registers an output filter \var{handler} under name
\var{name}. \var{Handler} is a module name optionally followed
\code{::} and a callable object name. If callable object name is
omitted, it will default to \samp{outputfilter}. \var{Name} is the name under
which the filter is registered, by convention filter names are usually
in all caps.

To activate the filter, use the \code{AddOutputFilter} directive.

\section{Connection Handler\label{dir-conn}}

\subsection{PythonConnectionHandler\label{dir-conn-ch}}
\index{PythonConnectionHandler}

\strong{\citetitle[http://httpd.apache.org/docs-2.0/mod/directive-dict.html#Syntax]{Syntax:}}
PythonConnectionHandler handler\\
\citetitle[http://httpd.apache.org/docs-2.0/mod/directive-dict.html#Context]{Context:}
server config\\
\citetitle[http://httpd.apache.org/docs-2.0/mod/directive-dict.html#Module]{Module:}
mod_python.c

Specifies that the connection should be handled with \var{handler}
connection handler. \var{Handler} will be passed a single argument -
the connection object.

\var{Handler} is a module name optionally followed \code{::} and a
callable object name. If callable object name is omitted, it will
default to \samp{connectionhandler}.

\section{Other Directives\label{dir-other}}

\subsection{PythonEnablePdb\label{dir-other-epd}}
\index{PythonEnablePdb}

\strong{\citetitle[http://httpd.apache.org/docs-2.0/mod/directive-dict.html#Syntax]{Syntax:}}
PythonEnablePdb \{On, Off\} \\
\citetitle[http://httpd.apache.org/docs-2.0/mod/directive-dict.html#Default]{Default:}
PythonEnablePdb Off\\
\citetitle[http://httpd.apache.org/docs-2.0/mod/directive-dict.html#Context]{Context:}
server config, virtual host, directory, htaccess\\
\citetitle[http://httpd.apache.org/docs-2.0/mod/directive-dict.html#Override]{Override:}
not None\\
\citetitle[http://httpd.apache.org/docs-2.0/mod/directive-dict.html#Module]{Module:}
mod_python.c

When On, mod_python will execute the handler functions within the
Python debugger pdb using the \code{pdb.runcall()} function.

Because pdb is an interactive tool, start httpd from the command line
with the -DONE_PROCESS option when using this directive. As soon as
your handler code is entered, you will see a Pdb prompt allowing you
to step through the code and examine variables.

\subsection{PythonDebug\label{dir-other-pd}}
\index{PythonDebug}

\strong{\citetitle[http://httpd.apache.org/docs-2.0/mod/directive-dict.html#Syntax]{Syntax:}}
PythonDebug \{On, Off\} \\
\citetitle[http://httpd.apache.org/docs-2.0/mod/directive-dict.html#Default]{Default:}
PythonDebug Off\\
\citetitle[http://httpd.apache.org/docs-2.0/mod/directive-dict.html#Context]{Context:}
server config, virtual host, directory, htaccess\\
\citetitle[http://httpd.apache.org/docs-2.0/mod/directive-dict.html#Override]{Override:}
not None\\
\citetitle[http://httpd.apache.org/docs-2.0/mod/directive-dict.html#Module]{Module:}
mod_python.c

Normally, the traceback output resulting from uncaught Python errors
is sent to the error log. With PythonDebug On directive specified, the
output will be sent to the client (as well as the log), except when
the error is \exception{IOError} while writing, in which case it will go
to the error log.

This directive is very useful during the development process. It is
recommended that you do not use it production environment as it may
reveal to the client unintended, possibly sensitive security
information.

\subsection{PythonImport\label{dir-other-pimp}}
\index{PythonImport}

\strong{\citetitle[http://httpd.apache.org/docs-2.0/mod/directive-dict.html#Syntax]{Syntax:}}
PythonImport \emph{module} \emph{interpreter_name}\\
\citetitle[http://httpd.apache.org/docs-2.0/mod/directive-dict.html#Context]{Context:}
server config\\
\citetitle[http://httpd.apache.org/docs-2.0/mod/directive-dict.html#Module]{Module:}
mod_python.c

Tells the server to import the Python module module at process startup
under the specified interpreter name. This is useful for
initialization tasks that could be time consuming and should not be
done at the request processing time, e.g. initializing a database
connection.

The import takes place at child process initialization, so the module
will actually be imported once for every child process spawned.

\begin{notice}
  At the time when the import takes place, the configuration is not
  completely read yet, so all other directives, including
  PythonInterpreter have no effect on the behavior of modules imported
  by this directive. Because of this limitation, the interpreter must
  be specified explicitly, and must match the name under which
  subsequent requests relying on this operation will execute. If you
  are not sure under what interpreter name a request is running,
  examine the \member{interpreter} member of the request object.
\end{notice}

See also Multiple Interpreters. 

\subsection{PythonInterpPerDirectory\label{dir-other-ipd}}
\index{PythonInterpPerDirectory}

\strong{\citetitle[http://httpd.apache.org/docs-2.0/mod/directive-dict.html#Syntax]{Syntax:}}
PythonInterpPerDirectory \{On, Off\} \\
\citetitle[http://httpd.apache.org/docs-2.0/mod/directive-dict.html#Default]{Default:}
PythonInterpPerDirectory Off\\
\citetitle[http://httpd.apache.org/docs-2.0/mod/directive-dict.html#Context]{Context:}
server config, virtual host, directory, htaccess\\
\citetitle[http://httpd.apache.org/docs-2.0/mod/directive-dict.html#Override]{Override:}
not None\\
\citetitle[http://httpd.apache.org/docs-2.0/mod/directive-dict.html#Module]{Module:}
mod_python.c

Instructs mod_python to name subinterpreters using the directory of
the file in the request (\code{req.filename}) rather than the the
server name. This means that scripts in different directories will
execute in different subinterpreters as opposed to the default policy
where scripts in the same virtual server execute in the same
subinterpreter, even if they are in different directories.

For example, assume there is a
\file{/directory/subdirectory}. \file{/directory} has an .htaccess
file with a PythonHandler directive.  \file{/directory/subdirectory}
doesn't have an .htaccess. By default, scripts in /directory and
\file{/directory/subdirectory} would execute in the same interpreter assuming
both directories are accessed via the same virtual server. With
PythonInterpPerDirectory, there would be two different interpreters,
one for each directory.

\begin{notice}
  In early phases of the request prior to the URI translation
  (PostReadRequestHandler and TransHandler) the path is not yet known
  because the URI has not been translated. During those phases and
  with PythonInterpPerDirectory on, all python code gets executed in
  the main interpreter. This may not be exactly what you want, but
  unfortunately there is no way around this.
\end{notice}

\begin{seealso}
  \seetitle[pyapi-interps.html]{Section \ref{pyapi-interps} Multiple Interpreters}
           {for more information}
\end{seealso}

\subsection{PythonInterpPerDirective\label{dir-other-ipdv}}
\index{PythonPythonInterpPerDirective}

\strong{\citetitle[http://httpd.apache.org/docs-2.0/mod/directive-dict.html#Syntax]{Syntax:}}
PythonInterpPerDirective \{On, Off\} \\
\citetitle[http://httpd.apache.org/docs-2.0/mod/directive-dict.html#Default]{Default:}
PythonInterpPerDirective Off\\
\citetitle[http://httpd.apache.org/docs-2.0/mod/directive-dict.html#Context]{Context:}
server config, virtual host, directory, htaccess\\
\citetitle[http://httpd.apache.org/docs-2.0/mod/directive-dict.html#Override]{Override:}
not None\\
\citetitle[http://httpd.apache.org/docs-2.0/mod/directive-dict.html#Module]{Module:}
mod_python.c

Instructs mod_python to name subinterpreters using the directory in
which the Python*Handler directive currently in effect was
encountered.

For example, assume there is a
\file{/directory/subdirectory}. \file{/directory} has an .htaccess
file with a PythonHandler directive.  \file{/directory/subdirectory}
has another \file{.htaccess} file with another PythonHandler. By
default, scripts in \file{/directory} and
\file{/directory/subdirectory} would execute in the same interpreter
assuming both directories are in the same virtual server. With
PythonInterpPerDirective, there would be two different interpreters,
one for each directive.

\begin{seealso}
  \seetitle[pyapi-interps.html]{Section \ref{pyapi-interps} Multiple Interpreters}
           {for more information}
\end{seealso}

\subsection{PythonInterpreter\label{dir-other-pi}}
\index{PythonInterpreter}

\strong{\citetitle[http://httpd.apache.org/docs-2.0/mod/directive-dict.html#Syntax]{Syntax:}}
PythonInterpreter name \\
\citetitle[http://httpd.apache.org/docs-2.0/mod/directive-dict.html#Context]{Context:}
server config, virtual host, directory, htaccess\\
\citetitle[http://httpd.apache.org/docs-2.0/mod/directive-dict.html#Override]{Override:}
not None\\
\citetitle[http://httpd.apache.org/docs-2.0/mod/directive-dict.html#Module]{Module:}
mod_python.c

Forces mod_python to use interpreter named \emph{name}, overriding the
default behaviour or behaviour dictated by
\citetitle[dir-other-ipd.html]{\code{PythonInterpPerDirectory}} or
\citetitle[dir-other-ipdv.html]{\code{PythonInterpPerDirective}} directive.

This directive can be used to force execution that would normally
occur in different subinterpreters to run in the same one. When
specified in the DocumentRoot, it forces the whole server to run in one
subinterpreter.

\begin{seealso}
  \seetitle[pyapi-interps.html]{Section \ref{pyapi-interps} Multiple Interpreters}
           {for more information}
\end{seealso}

\subsection{PythonHandlerModule\label{dir-other-phm}}
\index{PythonHandlerModule}

\strong{\citetitle[http://httpd.apache.org/docs-2.0/mod/directive-dict.html#Syntax]{Syntax:}}
PythonHandlerModule module \\
\citetitle[http://httpd.apache.org/docs-2.0/mod/directive-dict.html#Context]{Context:}
server config, virtual host, directory, htaccess\\
\citetitle[http://httpd.apache.org/docs-2.0/mod/directive-dict.html#Override]{Override:}
not None\\
\citetitle[http://httpd.apache.org/docs-2.0/mod/directive-dict.html#Module]{Module:}
mod_python.c

PythonHandlerModule can be used an alternative to Python*Handler
directives. The module specified in this handler will be searched for
existence of functions matching the default handler function names,
and if a function is found, it will be executed.

For example, instead of:
\begin{verbatim}
  PythonAuthenHandler mymodule
  PythonHandler mymodule
  PythonLogHandler mymodule
\end{verbatim}    

one can simply say
\begin{verbatim}
  PythonHandlerModule mymodule
\end{verbatim}    

\subsection{PythonAutoReload\label{dir-other-par}}
\index{PythonAutoReload}

\strong{\citetitle[http://httpd.apache.org/docs-2.0/mod/directive-dict.html#Syntax]{Syntax:}}
PythonAutoReload \{On, Off\} \\
\citetitle[http://httpd.apache.org/docs-2.0/mod/directive-dict.html#Default]{Default:}
PythonAutoReload On\\
\citetitle[http://httpd.apache.org/docs-2.0/mod/directive-dict.html#Context]{Context:}
server config, virtual host, directory, htaccess\\
\citetitle[http://httpd.apache.org/docs-2.0/mod/directive-dict.html#Override]{Override:}
not None\\
\citetitle[http://httpd.apache.org/docs-2.0/mod/directive-dict.html#Module]{Module:}
mod_python.c

If set to Off, instructs mod_python not to check the modification date
of the module file. 

By default, mod_python checks the time-stamp of the file and reloads
the module if the module's file modification date is later than the
last import or reload. This way changed modules get automatically
reimported, eliminating the need to restart the server for every
change.

Disabling autoreload is useful in production environment where the
modules do not change; it will save some processing time and give a
small performance gain.

\subsection{PythonOptimize\label{dir-other-pomz}}
\index{PythonOptimize}

\strong{\citetitle[http://httpd.apache.org/docs-2.0/mod/directive-dict.html#Syntax]{Syntax:}}
PythonOptimize \{On, Off\} \\
\citetitle[http://httpd.apache.org/docs-2.0/mod/directive-dict.html#Default]{Default:}
PythonOptimize Off\\
\citetitle[http://httpd.apache.org/docs-2.0/mod/directive-dict.html#Context]{Context:}
server config\\
\citetitle[http://httpd.apache.org/docs-2.0/mod/directive-dict.html#Module]{Module:}
mod_python.c

Enables Python optimization. Same as the Python \programopt{-O} option.

\subsection{PythonOption\label{dir-other-po}}
\index{PythonOption}

\strong{\citetitle[http://httpd.apache.org/docs-2.0/mod/directive-dict.html#Syntax]{Syntax:}}
PythonOption key [value] \\
\citetitle[http://httpd.apache.org/docs-2.0/mod/directive-dict.html#Context]{Context:}
server config, virtual host, directory, htaccess\\
\citetitle[http://httpd.apache.org/docs-2.0/mod/directive-dict.html#Override]{Override:}
not None\\
\citetitle[http://httpd.apache.org/docs-2.0/mod/directive-dict.html#Module]{Module:}
mod_python.c

Assigns a key value pair to a table that can be later retrieved by the
\code{req.get_options()} function. This is useful to pass information
between the apache configuration files (\file{httpd.conf},
\file{.htaccess}, etc) and the Python programs. If the value is omitted or empty (\code{""}),
then the key is removed from the local configuration.

\subsection{PythonPath\label{dir-other-pp}}
\index{PythonPath}

\strong{\citetitle[http://httpd.apache.org/docs-2.0/mod/directive-dict.html#Syntax]{Syntax:}}
PythonPath \emph{path} \\
\citetitle[http://httpd.apache.org/docs-2.0/mod/directive-dict.html#Context]{Context:}
server config, virtual host, directory, htaccess\\
\citetitle[http://httpd.apache.org/docs-2.0/mod/directive-dict.html#Override]{Override:}
not None\\
\citetitle[http://httpd.apache.org/docs-2.0/mod/directive-dict.html#Module]{Module:}
mod_python.c

PythonPath directive sets the PythonPath. The path must be specified
in Python list notation, e.g.

\begin{verbatim}
  PythonPath "['/usr/local/lib/python2.0', '/usr/local/lib/site_python', '/some/other/place']"
\end{verbatim}

The path specified in this directive will replace the path, not add to
it. However, because the value of the directive is evaled, to append a
directory to the path, one can specify something like

\begin{verbatim}
  PythonPath "sys.path+['/mydir']"
\end{verbatim}

Mod_python tries to minimize the number of evals associated with the
PythonPath directive because evals are slow and can negatively impact
performance, especially when the directive is specified in an
\file{.htaccess} file which gets parsed at every hit. Mod_python will
remember the arguments to the PythonPath directive in the un-evaled
form, and before evaling the value it will compare it to the
remembered value. If the value is the same, no action is
taken. Because of this, you should not rely on the directive as a way
to restore the pythonpath to some value if your code changes it.

\begin{notice}
  This directive should not be used as a security measure since the
  Python path is easily manipulated from within the scripts.
\end{notice}
