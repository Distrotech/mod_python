\chapter{Installation\label{installation}}
\indexii{installation}{UNIX}
\indexii{mod_python}{mailing list}
NOTE: By far the best place to get help with installation and other
issues is the mod_python mailing list. Please take a moment to join
the mod_python mailing list by sending an e-mail with the word
"subscribe" in the subject to \email{mod_python-request@modpython.org}.

\section{Prerequisites\label{inst-prerequisites}}

\begin{itemize}
\item
Python 2.2. Later versions and versions as old as 1.5.2 should work as well.
\item
Apache 2.0. (For Apache 1.3.x, use mod_python version 2.7.x)
\end{itemize}

You will need to have the include files for both Apache and Python, as
well as the Python library installed on your system.  If you installed
Python and Apache from source, then you already have everything that's
needed. However, if you are using prepackaged software (e.g. Linux Red
Hat RPM, Debian, or Solaris packages from sunsite, etc) then chances
are, you have just the binaries and not the sources on your
system. Often, the Apache and Python include files and libraries
necessary to compile mod_python are part of separate "development"
package. If you are not sure whether you have all the necessary files,
either compile and install Python and Apache from source, or refer to
the documentation for your system on how to get the development
packages.

\section{Compiling\label{inst-compiling}}
\indexii{compiling}{mod_python}

There are two ways in which modules can be compiled and linked to
Apache - statically, or as a DSO (Dynamic Shared Object).

\dfn{Static} linking is an older approach. With dynamic linking
available on most platforms it is used less and less. The main
drawback is that it entails recompiling Apache, which some people
cannot do for a variety of reasons.

\dfn{DSO} is a more popular approach nowadays and is the recommended
one for mod_python. The module gets compiled as a shared library that
is dynamically loaded by the server at run time. 

The advantage of DSO is that a module can be installed without
recompiling Apache and used as needed.  A more detailed description of
the Apache DSO mechanism is available at
\url{http://httpd.apache.org/docs-2.0/dso.html}.

At this time only DSO is supported by mod_python.

\subsection{Running ./configure\label{inst-configure}}
\index{./configure}

The \program{./configure} script will analyze your environment and create custom
Makefiles particular to your system. Aside from all the standard
autoconf stuff, \program{./configure} does the following:

\begin{itemize}

\item
\index{apxs}
\indexii{./configure}{\longprogramopt{with-apxs}}
Finds out whether a program called \program{apxs} is available. This
program is part of the standard Apache distribution, and is necessary
for DSO compilation. If apxs cannot be found in your \envvar{PATH} or in
\filenq{/usr/local/apache/bin}, DSO compilation will not be available.

You can manually specify the location of apxs by using the
\longprogramopt{with-apxs} option, e.g.:

\begin{verbatim}
$ ./configure --with-apxs=/usr/local/apache/bin/apxs 	
\end{verbatim}
%$ keep emacs happy

It is strongly recommended that you do specify this option.

%\item
%\indexii{./configure}{\longprogramopt{with-apache}}
%Checks for \longprogramopt{with-apache} option. Use this option to
%tell \program{./configure} where the Apache sources are on your
%system. The Apache sources are necessary for static compilation. If
%you do not specify this option, static compilation will not be
%available. Here is an example:

%\begin{verbatim}
%$ ./configure --with-apache=../src/apache_1.3.12 --with-apxs=/usr/local/apache/bin/apxs
%\end{verbatim}
%%$ keep emacs happy

\item
\index{libpython.a}
Checks your Python version and attempts to figure out where
\program{libpython} is by looking at various parameters compiled into
your Python binary. By default, it will use the \program{python}
program found in your \envvar{PATH}.

\indexii{./configure}{\longprogramopt{with-python}} If the Python
installation on your system is not suitable or not the one desired for
mod_python, you can specify an alternative location with the
\longprogramopt{with-python} options. This option needs to point to
the root directory of the Python source, e.g.:

\begin{verbatim}
$ ./configure --with-python=/usr/local/src/Python-2.2.1
\end{verbatim}                      
%$ keep emacs happy

Note that the directory specified needs to contain already configured and
compiled Python. In other words, you must at least run \program{./configure} and
\program{make} in this directory.

Also note that while it is possible to point the
\longprogramopt{with-python} to a version of Python different from the
one installed in your standard \envvar{PATH}, you will need to have
that version of Python installed as well.  This is because the path to
the Python library, which is retrieved from the \file{python} binary
is going to point to the place where Python would be ultimately
installed, not the source deirectory. (Consider that until Python is
installed, the location for third party libraries,
\filenq{site-packages}, does not exist). Generally, it's best to try
to keep the version of Python that you use for mod_python the same as
the one you use everywhere on the system.

\end{itemize}

\subsection{Running make\label{inst-make}}

\begin{itemize}

\item
%If possible, the \program{./configure} script will default to DSO
%compilation, otherwise, it will default to static. To stay with
%whatever \program{./configure} decided, simply run
To start the build process, simply run
\begin{verbatim}
$ make
\end{verbatim}
%$ emacs happy

%\indexii{make targets}{static}
%\indexii{make targets}{dso}

%Or, if you would like to be specific, give \program{make} a
%\programopt{dso} or \programopt{static} target:
%\begin{verbatim}
%$ make dso
%\end{verbatim}
%%$ emacs happy
%OR
%\begin{verbatim}
%$ make static
%\end{verbatim}
%%$

\end{itemize}

\section{Installing\label{inst-installing}}

\subsection{Running make install\label{inst-makeinstall}}

\begin{itemize}

\item
This part of the installation needs to be done as root. 
\begin{verbatim}
$ su
# make install
\end{verbatim}
%$ emacs happy
                      
\begin{itemize}

\item
%For DSO, this will simply copy the library into your Apache \filenq{libexec}
This will simply copy the library into your Apache \filenq{libexec}
directory, where all the other modules are.

%\item
%For static, it will copy some files into your Apache source tree.

\item
Lastly, it will install the Python libraries in \filenq{site-packages} and
compile them. 

\end{itemize} 

\indexii{make targets}{install_py_lib}
%\indexii{make targets}{install_static}
\indexii{make targets}{install_dso}
\strong{NB:} If you wish to selectively install just the Python libraries
%the static library or the DSO (which may not always require superuser
or the DSO (which may not always require superuser
privileges), you can use the following \program{make} targets:
\programopt{install_py_lib} and \programopt{install_dso}
%\programopt{install_static} and \programopt{install_dso}

\end{itemize}

\subsection{Configuring Apache\label{inst-apacheconfig}}

\begin{itemize}

\item
If you compiled mod_python as a DSO, you will need to tell Apache to
load the module by adding the following line in the Apache
configuration file, usually called \filenq{httpd.conf} or
\filenq{apache.conf}:

\begin{verbatim}
LoadModule python_module libexec/mod_python.so
\end{verbatim}

\index{mod_python.so}
The actual path to \program{mod_python.so} may vary, but make install
should report at the very end exactly where \program{mod_python.so}
was placed and how the \code{LoadModule} directive should appear.

If your Apache configuration uses \code{ClearModuleList} directive,
you will need to add mod_python to the module list in the Apache
configuration file:

\begin{verbatim}
AddModule mod_python.c
\end{verbatim}

%\item
%If you used the static installation, you now need to recompile Apache: 

%\begin{verbatim}
%$ cd ../src/apache_1.3.12
%$ ./configure --activate-module=src/modules/python/libpython.a
%$ make
%\end{verbatim}
%%$ emacs happy
                            
%Or, if you prefer the old "Configure" style, edit
%\filenq{src/Configuration} to have

%\begin{verbatim}
%AddModule modules/python/libpython.a
%\end{verbatim}
%then run 
%\begin{verbatim}
%$ cd src
%$ ./Configure
%$ Make
%\end{verbatim}
%%$ emacs happy

\end{itemize}

\section{Testing\label{inst-testing}}

\begin{enumerate}

\item
Make some directory that would be visible on your web site, for
example, htdocs/test.

\item
Add the following Apache directives, which can appear in either the
main server configuration file, or \filenq{.htaccess}.  If you are going
to be using the \filenq{.htaccess} file, you will not need the
\code{<Directory>} tag below, and you will need to make sure the
\code{AllowOverride} directive applicable to this directory has at least
\code{FileInfo} specified. (The default is \code{None}, which will not work.)
% the above has  been verified to be still true for Apache 2.0

\begin{verbatim}
<Directory /some/directory/htdocs/test> 
  AddHandler python-program .py
  PythonHandler mptest 
  PythonDebug On 
</Directory>
\end{verbatim}

(Substitute \filenq{/some/directory} above for something applicable to
your system, usually your Apache ServerRoot)

\item
At this time, if you made changes to the main configuration file, you
will need to restart Apache in order for the changes to take effect.

\item
Edit \filenq{mptest.py} file in the \filenq{htdocs/test} directory so
that is has the following lines (be careful when cutting and pasting
from your browser, you may end up with incorrect indentation and a
syntax error):

\begin{verbatim}
from mod_python import apache

def handler(req):
    req.write("Hello World!")
    return apache.OK 
\end{verbatim}

\item
Point your browser to the URL referring to the \filenq{mptest.py}; you
should see \code{"Hello World!"}. If you didn't - refer to the
troubleshooting section next.

\item
If everything worked well, move on to Chapter \ref{tutorial}, 
\citetitle[tutorial.html]{Tutorial}. 

\end{enumerate}

\section{Troubleshooting\label{inst-trouble}}

There are a few things you can try to identify the problem: 

\begin{itemize}

\item Carefully study the error output, if any. 

\item Check the server error log file, it may contain useful clues. 

\item Try running Apache from the command line in single process mode:
\begin{verbatim}
                  ./httpd -DONE_PROCESS
\end{verbatim}
This prevents it from backgrounding itself and may provide some useful
information.

\item
Ask on the mod_python list. Make sure to provide specifics such as:

\begin{itemize}

\item Mod_python version.
\item Your operating system type, name and version.
\item Your Python version, and any unusual compilation options.
\item Your Apache version.
\item Relevant parts of the Apache config, .htaccess.
\item Relevant parts of the Python code.

\end{itemize}

\end{itemize}
